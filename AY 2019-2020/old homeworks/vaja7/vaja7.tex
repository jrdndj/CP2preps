\documentclass[a4paper,12pt]{article}
\usepackage[utf8]{inputenc}
\usepackage[T1]{fontenc}
\usepackage[slovene]{babel}

\title{Računalniški praktikum II}
\author{Naloga za samostojno reševanje 7}
\date{12. april 2019}

\begin{document}
\maketitle
\thispagestyle{empty}

\noindent
Datoteka \texttt{chess.html} s pomočjo tabele izrisuje šahovnico.
\begin{enumerate}
\item Statični izris šahovnice zamenjajte s PHP skripto, ki izriše prazno šahovnico. Potrebna bo dvojna \emph{for} zanka -- zunanja za izris vrstic in notranja za izris stolpcev. Ne spreglejte, da so tudi oznake $A$, $B$, $C$, $1$, $2$, $3$, \ldots izrisane v poljih tabele.

\item Dodajte možnost postavitve poljubne figure na polje \texttt{A2} preko metode GET. Npr.:
\begin{verbatim}
chess.php?A2=ckralj
\end{verbatim}
na polje \texttt{A2} postavi črnega kralja. Vrednost parametra \texttt{A2} povežite z ustrezno sliko figure. Slike figur dobite na Wikimediji (povezave so podane spodaj). Črnega kralja lahko torej izrišemo na sledeči način:
\begin{verbatim}
<img src="http://upload.wikimedia.org/wikipedia/
          commons/e/e3/Chess_kdt60.png">
\end{verbatim}

\item Dodajte možnost postavitve poljubnih figur na poljubna polja, npr.:
\begin{verbatim}
chess.php?A2=ckralj&D4=bkmet&F6=bkraljica
\end{verbatim}
na polje \texttt{A2} postavi črnega kralja, na \texttt{D4} belega kmeta, na \texttt{F6} pa belo kraljico. Spodnji primer prikazuje možen način branja vhodnih parametrov.

\begin{verbatim}
for ($row = 1; $row <= 8; $row++) {
    for ($col = 'A'; $col <= 'H'; $col++) {
        $field = $col . $row;
        if (isset($_GET[$field]))
            echo "<div>Na polju $field je $_GET[$field]</div>\n";
    }
}
\end{verbatim}
\end{enumerate}

\section*{Slike šahovskih figur}
\texttt{http://upload.wikimedia.org/wikipedia/commons}

\medskip\noindent
\textbf{Bele figure}:

\begin{tabular}{ll}
\textit{kmet} & \texttt{/0/04/Chess\_plt60.png} \\
\textit{lovec} & \texttt{/9/9b/Chess\_blt60.png} \\
\textit{konj} & \texttt{/2/28/Chess\_nlt60.png} \\
\textit{trdnjava} & \texttt{/5/5c/Chess\_rlt60.png} \\
\textit{kralj} & \texttt{/3/3b/Chess\_klt60.png} \\
\textit{kraljica} & \texttt{/4/49/Chess\_qlt60.png}
\end{tabular}

\medskip\noindent
\textbf{Črne figure}:

\begin{tabular}{ll}
\textit{kmet} & \texttt{/c/cd/Chess\_pdt60.png} \\
\textit{lovec} & \texttt{/8/81/Chess\_bdt60.png} \\
\textit{konj} & \texttt{/f/f1/Chess\_ndt60.png} \\
\textit{trdnjava} & \texttt{/a/a0/Chess\_rdt60.png} \\
\textit{kralj} & \texttt{/e/e3/Chess\_kdt60.png} \\
\textit{kraljica} & \texttt{/a/af/Chess\_qdt60.png}
\end{tabular}

\end{document}